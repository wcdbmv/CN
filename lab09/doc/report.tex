\documentclass[a4paper,oneside,12pt]{extreport}

\include{preamble}

\begin{document}

\include{title}

\tableofcontents

\chapter{Задание 1}

Назначить адреса подсетей:
\begin{enumerate}
	\item Подсеть 1: 192.168.6.0/24
	\item Подсеть 2: 192.168.7.0/24
	\item Подсеть 3: 192.168.8.0/24
\end{enumerate}

\section{Настройка}

На хостах были настроены адреса интерфейсов.

\begin{table}[H]
	\centering
	\begin{tabular}{|c|c|c|}
		\hline
		\textbf{Хост} & \textbf{Адрес} & \textbf{Маска} \\ \hline
		Server0       & 192.168.6.1    & 255.255.255.0  \\ \hline
		Server1       & 192.168.6.2    & 255.255.255.0  \\ \hline
		PC0           & 192.168.8.1    & 255.255.255.0  \\ \hline
		PC1           & 192.168.8.2    & 255.255.255.0  \\ \hline
		PC2           & 192.168.8.3    & 255.255.255.0  \\ \hline
		PC3           & 192.168.7.1    & 255.255.255.0  \\ \hline
		PC4           & 192.168.7.2    & 255.255.255.0  \\ \hline
	\end{tabular}
\end{table}

\chapter{Задание 2}

Настроить поддержку трёх виртуальных локальных сетей (VLan 10, 20, 30) на коммутаторе.

\section{Настройка}

\begin{lstlisting}[gobble=8, caption=Настройка коммутатора Switch0]
	Switch>en
	Switch#sh vl

	VLAN Name                             Status    Ports
	---- -------------------------------- --------- -------------------------------
	1    default                          active    Fa0/1, Fa0/2, Fa0/3, Fa0/4
	Fa0/5, Fa0/6, Fa0/7, Fa0/8
	Fa0/9, Fa0/10, Fa0/11, Fa0/12
	Fa0/13, Fa0/14, Fa0/15, Fa0/16
	Fa0/17, Fa0/18, Fa0/19, Fa0/20
	Fa0/21, Fa0/22, Fa0/23, Fa0/24
	Gig0/1, Gig0/2
	1002 fddi-default                     active
	1003 token-ring-default               active
	1004 fddinet-default                  active
	1005 trnet-default                    active

	VLAN Type  SAID       MTU   Parent RingNo BridgeNo Stp  BrdgMode Trans1 Trans2
	---- ----- ---------- ----- ------ ------ -------- ---- -------- ------ ------
	1    enet  100001     1500  -      -      -        -    -        0      0
	1002 fddi  101002     1500  -      -      -        -    -        0      0
	1003 tr    101003     1500  -      -      -        -    -        0      0
	1004 fdnet 101004     1500  -      -      -        ieee -        0      0
	1005 trnet 101005     1500  -      -      -        ibm  -        0      0

	VLAN Type  SAID       MTU   Parent RingNo BridgeNo Stp  BrdgMode Trans1 Trans2
	---- ----- ---------- ----- ------ ------ -------- ---- -------- ------ ------

	Remote SPAN VLANs
	------------------------------------------------------------------------------

	Primary Secondary Type              Ports
	------- --------- ----------------- ------------------------------------------
	Switch#conf t
	Switch(config)#in v 10
	Switch(config-if)#interface range f0/1-f0/2
	Switch(config-if-range)#sw m a
	Switch(config-if-range)#sw a v 10
	Switch(config-if-range)#ex
	Switch(config)#in v 20
	Switch(config-if)#interface range f0/3-f0/4
	Switch(config-if-range)#sw m a
	Switch(config-if-range)#sw a v 20
	Switch(config-if-range)#ex
	Switch(config)#in v 30
	Switch(config-if)#interface range f0/5-f0/7
	Switch(config-if-range)#sw m a
	Switch(config-if-range)#sw a v 30
	Switch(config-if-range)#ex
	Switch(config)#ex
	Switch#sh vl

	VLAN Name                             Status    Ports
	---- -------------------------------- --------- -------------------------------
	1    default                          active    Fa0/8, Fa0/9, Fa0/10, Fa0/11
	Fa0/12, Fa0/13, Fa0/14, Fa0/15
	Fa0/16, Fa0/17, Fa0/18, Fa0/19
	Fa0/20, Fa0/21, Fa0/22, Fa0/23
	Fa0/24, Gig0/1, Gig0/2
	10   VLAN0010                         active    Fa0/1, Fa0/2
	20   VLAN0020                         active    Fa0/3, Fa0/4
	30   VLAN0030                         active    Fa0/5, Fa0/6, Fa0/7
	1002 fddi-default                     active
	1003 token-ring-default               active
	1004 fddinet-default                  active
	1005 trnet-default                    active

	VLAN Type  SAID       MTU   Parent RingNo BridgeNo Stp  BrdgMode Trans1 Trans2
	---- ----- ---------- ----- ------ ------ -------- ---- -------- ------ ------
	1    enet  100001     1500  -      -      -        -    -        0      0
	10   enet  100010     1500  -      -      -        -    -        0      0
	20   enet  100020     1500  -      -      -        -    -        0      0
	30   enet  100030     1500  -      -      -        -    -        0      0
	1002 fddi  101002     1500  -      -      -        -    -        0      0
	1003 tr    101003     1500  -      -      -        -    -        0      0
	1004 fdnet 101004     1500  -      -      -        ieee -        0      0
	1005 trnet 101005     1500  -      -      -        ibm  -        0      0

	VLAN Type  SAID       MTU   Parent RingNo BridgeNo Stp  BrdgMode Trans1 Trans2
	---- ----- ---------- ----- ------ ------ -------- ---- -------- ------ ------

	Remote SPAN VLANs
	------------------------------------------------------------------------------

	Primary Secondary Type              Ports
	------- --------- ----------------- ------------------------------------------
	Switch#ex
\end{lstlisting}

\chapter{Задание 3}

Настроить маршрутизацию между виртуальными локальными сетями на маршрутизаторе.

\section{Настройка}

\begin{lstlisting}[gobble=8, caption=Настройка маршрутизатора Router0]
	Router>en
	Router#conf t
	Router(config)#in g0/0/0.1
	Router(config-subif)#en d 10
	Router(config-subif)#ex
	Router(config)#in g0/0/0.2
	Router(config-subif)#en d 20
	Router(config-subif)#ex
	Router(config)#in g0/0/0.3
	Router(config-subif)#en d 30
	Router(config-subif)#ex
	Router(config)#in g0/0/0
	Router(config-if)#no sh
	Router(config-if)#ex
	Router(config)#ex
	Router#ex
\end{lstlisting}

\chapter{Задание 4}

Выделить и озаглавить на схеме каждую виртуальную локальную сеть.

\section{Результат}

\begin{figure}[H]
	\centering
	\includegraphics[width=0.7\linewidth]{inc/img/result.png}
\end{figure}

\end{document}
