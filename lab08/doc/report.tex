\documentclass[a4paper,oneside,12pt]{extreport}

\include{preamble}

\begin{document}

\include{title}

\tableofcontents

\chapter{Задание 1}

Назначить адреса подсетей:
\begin{enumerate}
	\item Подсеть 1: 192.168.6.0/24
	\item Подсеть 2: 192.168.7.0/24
	\item Подсеть 3: 192.168.8.0/24
	\item Подсеть 4: 192.168.9.0/24
	\item Подсеть 5 (В задаче III): 192.168.16.0/24
\end{enumerate}

\section{Настройка}

Настройка маршрутизаторов:

\begin{lstlisting}[gobble=8, caption=Настройка маршрутизатора Router0]
	Router>en
	Router#conf t
	Router(config)#in G 0/0/0
	Router(config-if)#ip ad 192.168.6.254 255.255.255.0
	Router(config-if)#ex
	Router(config)#in S 0/1/0
	Router(config-if)#ip ad 192.168.8.254 255.255.255.0
	Router(config-if)#ex
	Router(config)#ex
	Router#ex
\end{lstlisting}

\begin{lstlisting}[gobble=8, caption=Настройка маршрутизатора Router1]
	Router>en
	Router#conf t
	Router(config)#in G 0/0/0
	Router(config-if)#ip ad 192.168.7.254 255.255.255.0
	Router(config-if)#ex
	Router(config)#in S 0/1/0
	Router(config-if)#ip ad 192.168.9.254 255.255.255.0
	Router(config-if)#ex
	Router(config)#ex
	Router#ex
\end{lstlisting}

\begin{lstlisting}[gobble=8, caption=Настройка маршрутизатора Router2]
	Router>en
	Router#conf t
	Router(config)#in S 0/1/0
	Router(config-if)#ip ad 192.168.8.253 255.255.255.0
	Router(config-if)#ex
	Router(config)#in S 0/1/1
	Router(config-if)#ip ad 192.168.9.253 255.255.255.0
	Router(config-if)#ex
	Router(config)#ex
	Router#ex
\end{lstlisting}

\begin{lstlisting}[gobble=8, caption=Настройка маршрутизатора Router7]
	Router>en
	Router#conf t
	Router(config)#in G 0/0/0
	Router(config-if)#ip ad 192.168.6.254 255.255.255.0
	Router(config-if)#no sh
	Router(config-if)#ex
	Router(config)#in G 0/0/1
	Router(config-if)#ip ad 192.168.16.254 255.255.255.0
	Router(config-if)#no sh
	Router(config-if)#ex
	Router(config)#ex
	Router#ex
\end{lstlisting}

\begin{lstlisting}[gobble=8, caption=Настройка маршрутизатора Router8]
	Router>en
	Router#conf t
	Router(config)#in G 0/0/0
	Router(config-if)#ip ad 192.168.7.254 255.255.255.0
	Router(config-if)#no sh
	Router(config-if)#ex
	Router(config)#in G 0/0/1
	Router(config-if)#ip ad 192.168.16.253 255.255.255.0
	Router(config-if)#no sh
	Router(config-if)#ex
	Router(config)#ex
	Router#ex
\end{lstlisting}

\begin{lstlisting}[gobble=8, caption=Настройка маршрутизатора Router9]
	Router>en
	Router#conf t
	Router(config)#in G 0/0/0
	Router(config-if)#ip ad 192.168.8.254 255.255.255.0
	Router(config-if)#no sh
	Router(config-if)#ex
	Router(config)#in G 0/0/1
	Router(config-if)#ip ad 192.168.16.252 255.255.255.0
	Router(config-if)#no sh
	Router(config-if)#ex
	Router(config)#ex
	Router#ex
\end{lstlisting}

\begin{lstlisting}[gobble=8, caption=Настройка маршрутизатора Router10]
	Router>en
	Router#conf t
	Router(config)#in G 0/0/0
	Router(config-if)#ip ad 192.168.9.254 255.255.255.0
	Router(config-if)#no sh
	Router(config-if)#ex
	Router(config)#in G 0/0/1
	Router(config-if)#ip ad 192.168.16.251 255.255.255.0
	Router(config-if)#no sh
	Router(config-if)#ex
	Router(config)#ex
	Router#ex
\end{lstlisting}

На хостах были настроены адреса интерфейсов и адреса шлюзов по умолчанию.

\chapter{Задание 2}

Настроить динамическую маршрутизацию в прилагаемом .pkt файле на стенде I через протокол RIPv2 так, чтобы пинг любым хостом или маршрутизатором любого другого хоста или маршрутизатора был успешным.
Представить отдельным .pkt файлом.

\section{Настройка}

\begin{lstlisting}[gobble=8, caption=Настройка маршрутизатора Router0]
	Router>en
	Router#sh ip p
	Router#sh ip ri d
	Router#conf t
	Router(config)#ro r
	Router(config-router)#ne 192.168.6.0
	Router(config-router)#ne 192.168.8.0
	Router(config-router)#v 2
	Router(config-router)#ex
	Router(config)#ex
	Router#ex
\end{lstlisting}

\begin{lstlisting}[gobble=8, caption=Настройка маршрутизатора Router1]
	Router>en
	Router#sh ip p
	Router#sh ip ri d
	Router#conf t
	Router(config)#ro r
	Router(config-router)#ne 192.168.7.0
	Router(config-router)#ne 192.168.9.0
	Router(config-router)#v 2
	Router(config-router)#ex
	Router(config)#ex
	Router#ex
\end{lstlisting}

\begin{lstlisting}[gobble=8, caption=Настройка маршрутизатора Router2]
	Router>en
	Router#sh ip p
	Router#sh ip ri d
	Router#conf t
	Router(config)#ro r
	Router(config-router)#ne 192.168.8.0
	Router(config-router)#ne 192.168.9.0
	Router(config-router)#v 2
	Router(config-router)#ex
	Router(config)#ex
	Router#ex
\end{lstlisting}

\section{Проверка}

\begin{lstlisting}[gobble=8, caption=\code{Router0\# show ip route}]
	Router>en
	Router#sh ip ro
	Codes: L - local, C - connected, S - static, R - RIP, M - mobile, B - BGP
	D - EIGRP, EX - EIGRP external, O - OSPF, IA - OSPF inter area
	N1 - OSPF NSSA external type 1, N2 - OSPF NSSA external type 2
	E1 - OSPF external type 1, E2 - OSPF external type 2, E - EGP
	i - IS-IS, L1 - IS-IS level-1, L2 - IS-IS level-2, ia - IS-IS inter area
	* - candidate default, U - per-user static route, o - ODR
	P - periodic downloaded static route

	Gateway of last resort is not set

	192.168.6.0/24 is variably subnetted, 2 subnets, 2 masks
	C       192.168.6.0/24 is directly connected, GigabitEthernet0/0/0
	L       192.168.6.254/32 is directly connected, GigabitEthernet0/0/0
	R    192.168.7.0/24 [120/2] via 192.168.8.253, 00:00:13, Serial0/1/0
	192.168.8.0/24 is variably subnetted, 2 subnets, 2 masks
	C       192.168.8.0/24 is directly connected, Serial0/1/0
	L       192.168.8.254/32 is directly connected, Serial0/1/0
	R    192.168.9.0/24 [120/1] via 192.168.8.253, 00:00:13, Serial0/1/0

	Router#ex
\end{lstlisting}

\begin{figure}[H]
	\centering
	\includegraphics[width=0.495\linewidth]{inc/img/ping-0-3.png}
	\includegraphics[width=0.495\linewidth]{inc/img/ping-3-0.png}
	\caption{Пинги между PC0 и PC3}
\end{figure}

\chapter{Задание 3}

Настроить динамическую маршрутизацию в сети в прилагаемом .pkt файле на стенде II через протокол OSPF так, чтобы пинг любым хостом или маршрутизатором любого другого хоста или маршрутизатора был успешным.
Разделить при этом сеть на области OSPF в соответствии со схемой.
Выполнить указания в лабораторной работе.
Представить отдельным .pkt файлом.

\section{Настройка}

\begin{lstlisting}[gobble=8, caption=Настройка маршрутизатора Router7]
	Router>en
	Router#sh ip ospf interface

	Router#conf t
	Router(config)#route ospf 1
	Router(config-router)#network 192.168.6.0 0.0.0.255 area 1
	Router(config-router)#network 192.168.16.0 0.0.0.255 area 0
	Router(config-router)#ex
	Router(config)#in G 0/0/0
	Router(config-if)#ip ospf authentication-key key
	Router(config-if)#ex
	Router(config)#in G 0/0/1
	Router(config-if)#ip ospf authentication-key key
	Router(config-if)#ex
	Router(config)#ex
	Router#ex
\end{lstlisting}

\begin{lstlisting}[gobble=8, caption=Настройка маршрутизатора Router8]
	Router>en
	Router#sh ip ospf interface

	Router#conf t
	Router(config)#route ospf 1
	Router(config-router)#network 192.168.7.0 0.0.0.255 area 2
	Router(config-router)#network 192.168.16.0 0.0.0.255 area 0
	Router(config-router)#ex
	Router(config)#in G 0/0/0
	Router(config-if)#ip ospf authentication-key key
	Router(config-if)#ex
	Router(config)#in G 0/0/1
	Router(config-if)#ip ospf authentication-key key
	Router(config-if)#ex
	Router(config)#ex
	Router#ex
\end{lstlisting}

\begin{lstlisting}[gobble=8, caption=Настройка маршрутизатора Router9]
	Router>en
	Router#sh ip ospf interface

	Router#conf t
	Router(config)#route ospf 1
	Router(config-router)#network 192.168.8.0 0.0.0.255 area 3
	Router(config-router)#network 192.168.16.0 0.0.0.255 area 0
	Router(config-router)#ex
	Router(config)#in G 0/0/0
	Router(config-if)#ip ospf authentication-key key
	Router(config-if)#ex
	Router(config)#in G 0/0/1
	Router(config-if)#ip ospf authentication-key key
	Router(config-if)#ex
	Router(config)#ex
	Router#ex
\end{lstlisting}

\begin{lstlisting}[gobble=8, caption=Настройка маршрутизатора Router10]
	Router>en
	Router#sh ip ospf interface

	Router#conf t
	Router(config)#route ospf 1
	Router(config-router)#network 192.168.9.0 0.0.0.255 area 4
	Router(config-router)#network 192.168.16.0 0.0.0.255 area 0
	Router(config-router)#ex
	Router(config)#in G 0/0/0
	Router(config-if)#ip ospf authentication-key key
	Router(config-if)#ex
	Router(config)#in G 0/0/1
	Router(config-if)#ip ospf authentication-key key
	Router(config-if)#ex
	Router(config)#ex
	Router#ex
\end{lstlisting}

\section{Проверка}

\begin{lstlisting}[gobble=8, caption=\code{Router7\# show ip route}]
	Router>en
	Router#sh ip ro
	Codes: L - local, C - connected, S - static, R - RIP, M - mobile, B - BGP
	D - EIGRP, EX - EIGRP external, O - OSPF, IA - OSPF inter area
	N1 - OSPF NSSA external type 1, N2 - OSPF NSSA external type 2
	E1 - OSPF external type 1, E2 - OSPF external type 2, E - EGP
	i - IS-IS, L1 - IS-IS level-1, L2 - IS-IS level-2, ia - IS-IS inter area
	* - candidate default, U - per-user static route, o - ODR
	P - periodic downloaded static route

	Gateway of last resort is not set

	192.168.6.0/24 is variably subnetted, 2 subnets, 2 masks
	C       192.168.6.0/24 is directly connected, GigabitEthernet0/0/0
	L       192.168.6.254/32 is directly connected, GigabitEthernet0/0/0
	O IA 192.168.7.0/24 [110/2] via 192.168.16.253, 00:27:05, GigabitEthernet0/0/1
	O IA 192.168.8.0/24 [110/2] via 192.168.16.252, 00:27:05, GigabitEthernet0/0/1
	O IA 192.168.9.0/24 [110/2] via 192.168.16.251, 00:27:05, GigabitEthernet0/0/1
	192.168.16.0/24 is variably subnetted, 2 subnets, 2 masks
	C       192.168.16.0/24 is directly connected, GigabitEthernet0/0/1
	L       192.168.16.254/32 is directly connected, GigabitEthernet0/0/1

	Router#ex
\end{lstlisting}

\begin{figure}[H]
	\centering
	\includegraphics[width=0.495\linewidth]{inc/img/ping-7-10.png}
	\includegraphics[width=0.495\linewidth]{inc/img/ping-9-8.png}
	\caption{Пинги между (PC7 и PC10) и (PC9 и PC8)}
\end{figure}

\begin{lstlisting}[gobble=8, caption=\code{Router7\# show ip ospf neighbor}]
	Router>en
	Router#sh ip ospf neighbor
	Neighbor ID     Pri   State           Dead Time   Address         Interface
	192.168.16.251    1   FULL/DROTHER    00:00:37    192.168.16.251  GigabitEthernet0/0/1
	192.168.16.252    1   FULL/DROTHER    00:00:37    192.168.16.252  GigabitEthernet0/0/1
	192.168.16.253    1   FULL/BDR        00:00:37    192.168.16.253  GigabitEthernet0/0/1
	Router#ex
\end{lstlisting}

\begin{lstlisting}[gobble=8, caption=\code{Router8\#  show ip ospf neighbor}]
	Router>en
	Router#sh ip ospf neighbor
	Neighbor ID     Pri   State           Dead Time   Address         Interface
	192.168.16.251    1   FULL/DROTHER    00:00:31    192.168.16.251  GigabitEthernet0/0/1
	192.168.16.252    1   FULL/DROTHER    00:00:31    192.168.16.252  GigabitEthernet0/0/1
	192.168.16.254    1   FULL/DR         00:00:31    192.168.16.254  GigabitEthernet0/0/1
	Router#ex
\end{lstlisting}

\begin{lstlisting}[gobble=8, caption=\code{Router9\# show ip ospf neighbor}]
	Router>en
	Router#sh ip ospf neighbor
	Neighbor ID     Pri   State           Dead Time   Address         Interface
	192.168.16.251    1   2WAY/DROTHER    00:00:35    192.168.16.251  GigabitEthernet0/0/1
	192.168.16.253    1   FULL/BDR        00:00:36    192.168.16.253  GigabitEthernet0/0/1
	192.168.16.254    1   FULL/DR         00:00:35    192.168.16.254  GigabitEthernet0/0/1
	Router#ex
\end{lstlisting}

\begin{lstlisting}[gobble=8, caption=\code{Router10\#  show ip ospf neighbor}]
	Router>en
	Router#sh ip ospf neighbor
	Neighbor ID     Pri   State           Dead Time   Address         Interface
	192.168.16.252    1   2WAY/DROTHER    00:00:31    192.168.16.252  GigabitEthernet0/0/1
	192.168.16.253    1   FULL/BDR        00:00:31    192.168.16.253  GigabitEthernet0/0/1
	192.168.16.254    1   FULL/DR         00:00:30    192.168.16.254  GigabitEthernet0/0/1
	Router#ex
\end{lstlisting}

Router7 — DR.
Router8 — BDR.
Все маршрутизаторы являются пограничными.

\end{document}
